\documentclass[10pt]{article}
\usepackage{lmodern}
\usepackage{amssymb,amsmath}
% \usepackage{fontspec}
\usepackage[margin=1.15in]{geometry}
\usepackage{setspace, titling}
\newcommand{\subtitle}[1]{%
	\posttitle{%
		\par\end{center}
	\begin{center}\large#1\end{center}
	\vskip0.5em}%
}

%% FONTS
\usepackage{fontspec}
% See: https://tex.stackexchange.com/a/50593
\setmainfont[
BoldFont=FiraSans-Semibold.otf,
ItalicFont = FiraSans-Italic.otf,
BoldItalicFont = FiraSans-SemiBoldItalic.otf
]{FiraSans-Regular.otf} %
\setmonofont[
BoldFont = FiraCode-Bold.ttf
]{FiraCode-Regular.ttf}
\usepackage{marvosym} % For cool symbols.
\usepackage{fontawesome} % Ditto

\usepackage[normalem]{ulem} %% For strikeout font: \sout()

\usepackage[dvipsnames]{xcolor}
\definecolor{uo_green}{HTML}{154733}
\definecolor{forest_green}{HTML}{006241}
\definecolor{pine_green}{HTML}{007935}
\definecolor{grass_green}{HTML}{62A70F}
\definecolor{golden_yellow}{HTML}{FFD200}
\definecolor{cool_gray}{HTML}{54565B}
\definecolor{light_cool_gray}{HTML}{A8A8AA}

\usepackage[colorlinks = true,
linkcolor = pine_green,
urlcolor  = pine_green,
citecolor = pine_green,
anchorcolor = black]{hyperref}
\usepackage{graphicx}

% For table formatting:
\usepackage{array, booktabs, caption, siunitx}
\newcommand{\ra}[1]{\renewcommand{\arraystretch}{#1}}
\newcolumntype{d}[1]{D{.}{.}{#1}}

\begin{document}

\title{
	\texttt{\textbf{Introduction to Econometrics} [EC 320]}\\[1em]
	\large Winter 2022 Syllabus
}
\author{\textbf{Philip Economides} \\ Department of Economics \\ University of Oregon}
%\date{}  % Toggle commenting to test
\date{\vspace{-1ex}}

\maketitle

%\section*{Course at a glance}

\begin{table}[!h]
	\ra{1.1}
	\begin{tabular}{l @{\hspace{1.25\tabcolsep}} l l l @{\hspace{1.25\tabcolsep}} l l l @{\hspace{1.25\tabcolsep}} l @{}}
		& \textbf{{Lecture}} & & & \textbf{{Lab}} & & & \textbf{{Materials}} \\
		\faMapMarker & \href{https://map.uoregon.edu/55c9bd289}{Lillis 112} & & \faMapMarker &  \href{https://map.uoregon.edu/c959df181}{McKenzie 442} & & \faBook & \href{https://www.amazon.com/Mastering-Metrics-Path-Cause-Effect/dp/0691152845/}{Mastering `Metrics} \\
		\faClockO & MW, 1600--1720 & & \faClockO & See \href{https://duckweb.uoregon.edu/pls/prod/twbkwbis.P_WWWLogin}{DuckWeb} & & \faBook & \href{https://www.amazon.com/Introduction-Econometrics-Christopher-Dougherty/dp/0199676828/}{Introduction to Econometrics, 5\textsuperscript{th} ed. } \\
		& & & & & & \faLaptop & \href{https://www.r-project.org/}{\textbf{\texttt{R}}} \\
		& & & & & & \faLaptop & \href{https://www.rstudio.com}{\textbf{\texttt{RStudio}}}
	\end{tabular}
\end{table}

\begin{table}[!h]
	\ra{1.1}
	\begin{tabular}{l @{\hspace{1.25\tabcolsep}} l @{}}
		& \textbf{{Instructor}}\\
		\faUser & Philip Economides \\
		\faGlobe & \href{https://philip-economides.com/}{philip-economides.com} \\
		\faPaperPlaneO & \href{mailto:peconomi@uoregon.edu}{peconomi@uoregon.edu} \\
		\faMapMarker & \href{https://map.uoregon.edu/e99ccec73}{PLC 520} \\
		\faClockO & T 1500--1600, Th 1000--1100, or by appointment	
	\end{tabular}
\end{table}


\begin{table}[!h]
	\ra{1.1}
	\begin{tabular}{l @{\hspace{1.25\tabcolsep}} l @{}}
		& \textbf{{GE}}\\
		\faUser & Micaela Wood \\
		\faPaperPlaneO & \href{mailto:mwood13@uoregon.edu}{mwood13@uoregon.edu} \\
		\faMapMarker & \href{https://map.uoregon.edu/e062b003c}{PLC 827}\\
		\faClockO & F 1400--1500	 
	\end{tabular}
\end{table}

\section*{Course summary}

\paragraph{Description:} This course introduces the statistical techniques that help economists learn about the world using data. We will focus much of our attention on regression analysis, the workhorse of applied econometrics. Using calculus and introductory statistics, we will cultivate a working understanding of the theory underpinning regression analysis---\textit{how} it works, \textit{why} it works, and, \textit{when it can lead us astray}. We will apply the insights of theory to work with and learn from actual data using \texttt{{R}}, a statistical programming language. To the extent that you invest the requisite time and effort, you can leave this course with marketable skills in data analysis and---most importantly---a more sophisticated understanding of the notion that \textbf{correlation does not necessarily imply causation}. 

\paragraph{Prerequisites:} Math 242 (Calculus) and Math 243 (Introduction to Statistics) or equivalent.

\newpage

\subsection*{Software}

\begin{itemize}
	\item We will use the statistical programming language \href{https://www.r-project.org/}{\textbf{\texttt{R}}}.
	\item We will use \href{https://www.rstudio.com}{\textbf{\texttt{RStudio}}} to interact with \texttt{R}.
\end{itemize}
Learning \texttt{R} is challenging, but well worth the effort. \texttt{R} is a powerful and versatile tool for data analysis and visualization, which makes it popular among employers. If you dedicate the time and effort necessary to learn the language, you are likely to reap a handsome return on the job market. The SSIL lab in McKenzie has \texttt{R} and \texttt{RStudio} installed and ready for you, but I strongly recommend that you install these programs on your own computer. Don't worry, \textbf{both are free}. I also recommend that you save your scripts, data, and assignments separately from your preferred computing hardware. I'd suggest either using a portable flash drive or a Github repositories. Alternatively, you can use the class network drive (the ``R drive"), which is available on all university computers. I will be making material available through \href{https://github.com/peconomi/EC320_Econometrics}{\textbf{\texttt{Github}}} for convenient one-click downloading and those who lose their learning material thereafter. 

If you are concerned about learning \texttt{R}---or you want to learn quicker---I recommend that you check out the following free online resources:
\begin{itemize}
	\item \href{https://www.datacamp.com/courses/free-introduction-to-r}{DataCamp's \textit{Introduction to R}}
	\item \href{https://r4ds.had.co.nz/introduction.html}{Hadley Wickham's \textit{Introduction to R}}
\end{itemize}
The \texttt{RStudio} team has also assembled a \href{https://www.rstudio.com/online-learning/}{useful set of resources}.

\bigskip

\subsection*{Textbooks}

\paragraph{Required:} There is one required and one recommended textbook for this course:

\begin{enumerate}
	\item \href{http://www.amazon.com/Introduction-Econometrics-Christopher-Dougherty/dp/0199676828/}{\textbf{Introduction to Econometrics}, 5\textsuperscript{th} ed.} by Christopher Dougherty (\textbf{ItE}), required
	\item \href{https://www.amazon.com/Mastering-Metrics-Path-Cause-Effect/dp/0691152845/}{\textbf{Mastering `Metrics: The Path from Cause to Effect}} by Angrist and Pischke (\textbf{MM})
\end{enumerate}
You can purchase them at the Duckstore or your preferred online bookseller. You should complete the assigned readings from the textbooks \textit{before} lecture. Attending lecture is not a substitute for reading and comprehending the texts. Likewise, reading is not a substitute for attending lecture. The lectures and the readings are meant to \textit{complement} one another. The tentative course schedule (further below) lists the assigned readings for each topic.

In addition to the textbook readings, I may occasionally assign readings from peer-reviewed studies for classroom discussion. I will post these readings on Canvas.

\paragraph{Optional:} 
There is a wealth of free online books for learning \texttt{R}. 
A classic is Garrett Grolemund and Hadley Wickham's \href{http://r4ds.had.co.nz}{\textbf{\textit{R} for Data Science}}. 
If you have previous experience coding in \texttt{R}, you may want to check out Hadley Wickham's \href{http://adv-r.had.co.nz/}{\textbf{Advanced \textit{R}}}. 
If you are interested in producing beautiful and informative graphs or maps, see Kieran Healy's \href{http://socviz.co/}{\textbf{Data Visualization: A Practical Introduction}}. 
Our very own \href{https://grantmcdermott.com/teaching/}{\textbf{Grant McDermott}} and \href{http://edrub.in/teaching.html}{\textbf{Ed Rubin}} also maintain advanced resources for learning R.

\newpage
\section*{Course Structure}

\subsection*{Grades}

I will award grades based on your relative performance in the class, as determined by the following weights:
\begin{table}[!h]
	\ra{1.2}
	\centering
	\begin{tabular}{@{\extracolsep{1cm}}ll@{}}
		\textbf{Problem Sets} & 25\% \\
		\textbf{Quizzes} & 10\% \\
		\textbf{Midterm Exam} & 20\% \\
		\textbf{Data Assignment} & 10\%\\
		\textbf{Final Exam}   & 35\%
	\end{tabular}
\end{table}

\subsection*{Problem Sets} 

I will assign \textbf{five} bi-weekly problem sets throughout the quarter. Each problem set will include an analytical component and a computational component. 
\begin{itemize}
	\setlength{\itemsep}{0pt}
	\item I will announce due dates in class. 
	\item You will turn in an \textbf{electronic copy} of each problem set on Canvas.
	\item Presentation matters. The aim is to be ready to use your skills in practical settings. 10 percentage points of each grade will be attributed to producing professional work (\textit{e.g.}, clear language, typed equations, tasteful fonts, tables and graphs with informative labels, \textit{etc}.).
\end{itemize}
I encourage you to work together on the problem sets. 
Unless explicitly stated, \textbf{each student is required to write and submit independent answers}. 
I will take word-for-word and code-for-code copies as evidence of academic dishonesty. 
If you work with others, list their names at the top of your assignment. Groups must consist of {\bf three or less} individuals. 
If you fail to list your collaborators, you will receive a score of zero.

\subsection*{Quizzes}

I will assign \textbf{two} quizzes during the term, aimed at testing your knowledge on the recent content we've covered.
They'll be relatively easy compared to the homework, 8 questions per quiz, featuring MCQ's and true/false questions. 
I'll give 45mins for each quiz and several days to complete it. 
This maps closely to the recommendation that you read prescribed material \text{before class}. 
During the lecture, make sure you're asking questions about any particular topic, should the added explanation of material not be proving to be sufficient. 
These two quizzes will cover two separate portions of course content recently addressed in class.
Feel free to use your textbook throughout.
% Discuss length in terms of time and question count. 

\subsection*{Data Assignment}

I will assign a single data assignment to the class, in which I will make a number of separate datasets available.
These datasets will be available from the beginning of term, but I would recommend you do not commence work until after your lab on data wrangling with R's dplyr package.
If you wish to use a separate dataset or merging of datasets, email me in advanced to ensure it is an appropriate challenge for the class' requirements.
Upon completion your data assignment should include the following sections;
\begin{itemize}
	\setlength{\itemsep}{0pt}
	\item Introduction: Describe the data source type (panel, cross-sectional, etc.), its frequency (hourly, annual, etc.), and a bit of flavor about how representative the sample is relative to the population.
	\item Cleaning: Ultimately, you want to present this data in a clear and legible manner for its intended audience. If you are going to be performing regressions, this may require identifying outliers and treating for missing observations. 
	List the challenges you encountered when preparing this data for its intended audience. Outline some of the limitations the data may face upon tackling these challenges. 
	\item Summary Statistics: This is the most creative portion of the assignment. While you are not addressing causal inference in this portion of the assignment, your presentation of the data may still hint at potential research questions underlying the material. 
	Present a series of interesting facts about the data. Infer potential causal relationships and describe how you may go about testing those relationships.
	\item Regression: Provide inference based on a series of regressions carried out to explore your causal question of interest. Keep in mind that multiple stages of a regression are highly informative with respect to issues such as omitted variable bias.
	Discuss the key assumptions necessary in order for your results to be considered unbiased and inference with respect to standard errors to be apt. 
	\item Conclusion: For the more time-constraint reader, perhaps a future employer, you always want to be concise with your statements. Use this section to provide concluding remarks about the data, detailing how rich it appears to be and the potential uses it may provide in terms of empirical analysis.
\end{itemize}

\subsection*{Exams} 

For each exam, I will generate a randomized seating chart. 
During the exams, you may use a set of writing utensils, a non-programmable calculator, and a blank 3-by-5-inch notecard. 
I will have spare calculators and scrap paper available for anyone in need of additional resources. 
As you turn in your exam, I will ask you to present your student ID. 
I do not give makeup exams. 
See the course policy on makeup assignments for more information.

\bigskip 

\noindent Given the ongoing pandemic, should I observe any students experiencing profuse coughing during the exam, I reserve the choice to interrupt their exams immediately.
This otherwise becomes a major distraction for all other individuals taking their exam, and a particular safety hazard for any individuals unable to be vaccinated (e.g. auto-immune disorders, religious grounds, etc.).

\subsection*{Lab} 

In your weekly Thursday lab section, you will learn to apply the concepts discussed in lecture using \texttt{R}. 
While the lab may include some general econometrics instruction, the main focus is on the practical application of statistical techniques and working through the computational components of the bi-weekly problem sets. 
Attending lab is crucial for learning the material and passing the course. 
Everyone will have the option of either attending the {\bf 1600-1720} time slot at {\bf McKenzie 442}. Sessions will be recorded for those unable to attend. 
No lab will be ran for the {\bf 1730-1850} time slot. 

\newpage

\section*{Course Policies}

\subsection*{Late Policy} 

I will not accept late problem sets after the due date. If you turn in a problem set on the due date, but after the deadline, points will be deducted for lateness. If you turn it in after I post the key, you will receive a zero.

\subsection*{Makeup Assignments} 

I do not give makeup assignments. This blanket ban extends to exams. In extreme circumstances that lead you to miss one of the midterm exams---such as death in the family or grave illness or injury---I will consider re-weighting your grade toward the final. To qualify for re-weighting, you will need to notify me no later than two days after the exam.

\subsection*{Grade Appeals} 

You must submit any request for re-grading in writing within one week of the day grades are posted for the problem set or exam in question. Your request should include a cogent argument explaining why your responses warrant full credit.

\subsection*{Etiquette} 

Please respect those around you by turning off your phone and other potentially distracting devices. 
I ask that you stay for the entire lecture: getting up and leaving distracts your fellow classmates. 
If you must leave early, please position yourself near the door when you get to class. 
As a final note, a growing body of evidence suggests that \href{https://www.theverge.com/2017/11/27/16703904/laptop-learning-lecture}{using laptops in lecture reduces comprehension and recollection}. In light of this evidence, I ask that you refrain from using your laptop during lecture. 
As a practical matter, it is much easier to write math by hand than it is to type it. 

\subsection*{Academic Integrity} 

I will not tolerate cheating, plagiarism, and other violations of the \href{https://studentlife.uoregon.edu/conduct}{Student Conduct Code}. If you are caught cheating or plagiarizing on any component of this course, you will receive a failing grade for the term and I will report your offense to the university. 

\subsection*{Academic Disruption}

In the event of a campus emergency that disrupts academic activities, course requirements, deadlines, and grading percentages are subject to change. 
Information about changes in this course will be communicated as soon as possible by email, and on Canvas. 
If we are not able to meet face-to-face, students should immediately log onto Canvas and read any announcements and/or access alternative assignments. 
Students are also expected to continue coursework as outlined in this syllabus or other instructions on Canvas.

\bigskip

\noindent In the event that I am forced to quarantine, this course may be taught online during that time.

\subsection*{Accommodations} 

Notify me if there are aspects of this course that pose disability-related barriers to your participation. If you require special accommodations for a documented disability, then you will need to provide me a letter from the \href{https://aec.uoregon.edu/}{Accessible Education Center} (AEC) that verifies your need and details the appropriate accommodations. Please make arrangements with the AEC by the end of Week 1. If your accommodations include exam proctoring at the AEC, then you are responsible for scheduling those exams with the AEC \textit{at least seven days in advance}.

\subsection*{COVID Advice} 

I would ask that everyone tries their best to stick as closely to the \href{https://coronavirus.uoregon.edu/vaccine}{UO COVID-19 Vaccination Requirement} policy as possible. 
Additionally, I will be following \href{https://provost.uoregon.edu/academic-council-fall-2021-guidance-and-expectations-during-covid-19-pandemic}{academic guidance and expectations} to the letter. 
Regardless of whether you are opting in for the vaccination or weekly-test options, everyone is required to continue wearing masks indoors. 
Should I notice any individuals refusing to do so or repeatedly refusing to adjust their mask correctly, they will be asked to leave. 

\bigskip

\noindent I appreciate everyone's prolonged patience under these circumstances and will do my best to attend to any concerns regarding these requirements.
If you do have any concerns regarding these policies and my enforcement of them, please do not hesitate to get in touch with me by email or office hours. 


\newpage
\section*{Tentative Schedule}

\begin{table}[h!]
	\caption*{\large\textbf{Lectures and Exams}}
	\centering
	\ra{1.5}
	\begin{tabular}{@{\extracolsep{0.5cm}} c c l l @{}}
		\toprule
		\textbf{Week} & \textbf{Date} & \textbf{Topic} & \textbf{Reading}  \\ \toprule
		01 & 01/03 & Introduction & \\
		01 & 01/05 & Statistics Review I & ItE Review \\
		02 & 01/10 & Statistics Review II & ItE Review; MM 1 (appendix) \\
		02 & 01/12 & The Fundamental Econometric Problem & MM 1 \\
		03 & 01/19 & The Logic of Regression & MM 2  \\
		04 & 01/24 & Simple Linear Regression: Estimation I & ItE 1  \\ 
		04 & 01/26 & Simple Linear Regression: Estimation II & ItE 1\\ 
		05 & 01/31 & Classical Assumptions & ItE 2   \\
		05 & 02/02 & Simple Linear Regression: Inference & ItE 2 \\
		06 & 02/07 & Midterm Review & \\ \midrule 
		06 & 02/09 & \textbf{Midterm Exam} (in-class) & \\ \midrule
		07 & 02/14 & Multiple Linear Regression: Estimation & ItE 3, 6.2; MM 2 (appendix) \\
		07 & 02/16 & Multiple Linear Regression: Inference & ItE 3, 6.3; MM 2 (appendix) \\
		08 & 02/21 & Nonlinear Relationships  &  ItE 4  \\ 
		08 & 02/23 & Qualitative Variables & ItE 5 \\ 
		09 & 02/28 & Interactive Relationships & ItE 4  \\
		09 & 03/02 & Model Specification & ItE 6  \\
		10 & 03/07 & Heteroskedasticity &  ItE 7 \\
		10 & 03/09 & Final Review &   \\ \midrule
		11 & 03/16 & \textbf{Final Exam, 14:45} (see \href{https://registrar.uoregon.edu/calendars/examinations#complete-final-exam-schedule}{final exam schedule}) & \\
		\bottomrule
	\end{tabular}
\end{table}

\newpage

\begin{table}[h!]
	\caption*{\large\textbf{Labs, Quizzes and Problem Sets}}
	\centering
	\ra{1.5}
	\begin{tabular}{@{\extracolsep{0.5cm}} c c l @{}}
		\toprule
		\textbf{Week} & \textbf{Date} & \textbf{Topic}  \\ \toprule
		01 & 01/06 & Lab: Introduction to \texttt{R} and \texttt{RStudio} \\ 
		02 & 01/10 & \textbf{Problem Set 1} (due on Canvas by 5pm)  \\
		02 & 01/13 & Lab: Introduction to the \texttt{tidyverse} \\ \midrule 
		03 & 01/17 & \textbf{Quiz 1}: \textit{Basics} (due on Canvas by 11:59pm) \\
		03 & 01/20 & Lab: Visualization using \texttt{ggplot2} \\
		04 & 01/24 & \textbf{Problem Set 2} (due on Canvas by 5pm)  \\ 
		04 & 01/27 & Lab: Regression Analysis \\ 
		05 & 01/31 & \textbf{Problem Set 3} (due on Canvas by 5pm) \\ 
		05 & 02/03 & {\it No Lab}, focus on midterm prep \\  \midrule
		06 & 02/10 & Lab: Hypothesis Testing and Confidence Intervals \\
		07 & 02/18 & \textbf{Problem Set 4} (due on Canvas by 5pm)  \\ 
		07 & 02/17 & Lab: Omitted-Variable Bias Simulation \\
		08 & 02/23 & \textbf{Quiz 2}: \textit{Regressions} (due on Canvas by 11:59pm) \\ 
		08 & 02/24 & Lab: Maps on \texttt{R} + [taking requests] \\ \midrule
		09 & 03/01 & {\bf Data project}  \\  
		09 & 03/03 & Lab: Interaction terms and non-linear relationships \\
		10 & 03/07 & \textbf{Problem Set 5} (due on Canvas by 5pm) \\
		10 & 03/10 & Lab: Accounting for heteroskedasticity and autocorrelation \\ \bottomrule
	\end{tabular}
\end{table}


\end{document}